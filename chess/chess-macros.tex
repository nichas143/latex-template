
%% CHESS
\usepackage[T1]{fontenc}
\usepackage{chessboard}
\usepackage[ps]{skak}
\usepackage{xskak}
% set size of the board
%\setchessboard{boardfontsize=12pt,labelfontsize=10pt}



%\tcbuselibrary{listings,theorems,skins,breakable}
\usepackage{tikz}
\usepackage{varwidth}
\usepackage[utf8]{inputenc}
\usepackage[T1]{fontenc}
\usepackage[framemethod = tikz]{mdframed}






% defence 
\newcommand{\openingSecondMove}[3]{
	\subsection{#1 (1. #2 #3)}
	\newchessgame
	\hidemoves{1. #2}
	\mainline{1...#3}

	\chessboard[inverse, tinyboard]

}

% whites attack 
\newcommand{\openingThirdMove}[3]{
	\subsection{#1 ( #2 2. #3)}
	\newchessgame
	\hidemoves{#2}
	\mainline{2.#3}

	\chessboard[ tinyboard]

}

% defence 
\newcommand{\openingFourthMove}[3]{
	\subsection{#1 ( #2 #3)}
	\newchessgame
	\hidemoves{#2}
	\mainline{2...#3}

	\chessboard[inverse, tinyboard]

}

% whites attack 
\newcommand{\openingFifthMove}[3]{
	\subsection{#1 (#2 3. #3)}
	\newchessgame
	\hidemoves{#2}
	\mainline{3. #3}

	\chessboard[ tinyboard]

}

% defence 
\newcommand{\openingSixthMove}[3]{
	\subsection{#1 (#2 #3)}
	\newchessgame
	\hidemoves{#2}
	\mainline{3...#3}

	\chessboard[inverse, tinyboard]

}


%% THIS IS A NEW MACRO wrote on 18 Nov 2023
% It takes three arguments 
% 1. The move number 
% 2. The first move of white 
% 3. The first move of black. 
% \twomoves{<move no>}{e4}{e5} 

\newcommand{\tinychessboard}{\chessboard[tinyboard]}
\newcommand{\inversetinychessboard}{\chessboard[tinyboard,inverse]}


\newcommand{\twomoves}[3]{
\textbf{#1. #2 #3}

\ifthenelse{#1 = 1}{\newchessgame}{}
\hidemoves{#1. #2}
%\tinychessboard
%\inversetinychessboard
\hidemoves{#1...#3}
\tinychessboard
\inversetinychessboard
}

\newcommand{\onemove}[1]{
\textbf{#1}
\hidemoves{#1}

\tinychessboard
\inversetinychessboard
}



% SHOW MOVES (mark them) 
\newcommand{\showWhiteMoves}[1]{
\chessboard[pgfstyle = straightmove, color=red, markmoves={#1}]
}
\newcommand{\showBlackMoves}[1]{
\chessboard[pgfstyle = straightmove, color=red, backmoves={#1}]
}

\newcommand{\showMovesWithFen}[3]{
\chessboard[setfen=#1 #2 - - 0 0, pgfstyle = straightmove, color=red, markmoves={#3}]
}

\newcommand{\showtwomoves}[1]{
	\hidemoves{#1}
	\chessboard[tinyboard]
}